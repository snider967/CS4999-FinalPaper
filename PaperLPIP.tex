\documentclass[11pt]{report}

\begin{document}
\begin{Large}
Basic Definitions : Linear and Integer Programs \\
\end{Large}

Linear programming, or more specifically integer programming played a huge part in the motivation for work which was done. Therefore, in order for the rest of the paper to be fully understood some concepts from linear programming require definition. Linear programming problems occur when one wishes to maximize or minimize a linear function subject to some linear constraints. Observe the following example:\\

Find $z_1$ and $z_2$ which maximize the sum $z_1 + z_2$ subject to the following constraints:
$$z_1 \geq 0$$
$$z_2 \geq 0$$
$$2z_1 + 5z_2 \leq 22 $$
$$z_1 + z_2 \leq 6$$ 
$$3z_1 + z_2 \leq 17$$

As you can see, we have two unknowns, $z_1$ and $z_2$ which are bounded with five constraints. We call the special constraints of the form $z_i \geq 0$ nonnegativity constraints. The other three are known as the main constraints.  The function which is being minimized (or maximized) is known as the objective function and in this case is: $z_1 + z_2$. Any vector which satisfies all the constraints of a given linear program is a feasible solution. If a vector gives the minimum (or maximum) possible value out of all the feasible solutions it is known as the optimal solution or optimum. If a linear program has no feasible solutions, and therefore no optimum it is called infeasible. Linear programs are good because they're efficiently solvable in both theory and practice.
\end{document}
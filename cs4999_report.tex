% !TEX TS-program = pdflatex
% !TEX encoding = UTF-8 Unicode

% This is a simple template for a LaTeX document using the "article" class.
% See "book", "report", "letter" for other types of document.

\documentclass[11pt]{article} % use larger type; default would be 10pt

\usepackage[utf8]{inputenc} % set input encoding (not needed with XeLaTeX)

%%% Examples of Article customizations
% These packages are optional, depending whether you want the features they provide.
% See the LaTeX Companion or other references for full information.

%%% PAGE DIMENSIONS
\usepackage{geometry} % to change the page dimensions
\geometry{a4paper} % or letterpaper (US) or a5paper or....
% \geometry{margin=2in} % for example, change the margins to 2 inches all round
% \geometry{landscape} % set up the page for landscape
%   read geometry.pdf for detailed page layout information

\usepackage{graphicx} % support the \includegraphics command and options

% \usepackage[parfill]{parskip} % Activate to begin paragraphs with an empty line rather than an indent

%%% PACKAGES
\usepackage{booktabs} % for much better looking tables
\usepackage{amssymb}
\usepackage{amsthm}
\usepackage{array} % for better arrays (eg matrices) in maths
\usepackage{paralist} % very flexible & customisable lists (eg. enumerate/itemize, etc.)
\usepackage{verbatim} % adds environment for commenting out blocks of text & for better verbatim
\usepackage{subfig} % make it possible to include more than one captioned figure/table in a single float
% These packages are all incorporated in the memoir class to one degree or another...

%%% HEADERS & FOOTERS
\usepackage{fancyhdr} % This should be set AFTER setting up the page geometry
\pagestyle{fancy} % options: empty , plain , fancy
\renewcommand{\headrulewidth}{0pt} % customise the layout...
\lhead{}\chead{}\rhead{}
\lfoot{}\cfoot{\thepage}\rfoot{}

%%% SECTION TITLE APPEARANCE
\usepackage{relsize}
\usepackage{sectsty}
\allsectionsfont{\mdseries\upshape\larger} % (See the fntguide.pdf for font help)
% (This matches ConTeXt defaults)

%%% ToC (table of contents) APPEARANCE
\usepackage[nottoc,notlof,notlot]{tocbibind} % Put the bibliography in the ToC
\usepackage[titles,subfigure]{tocloft} % Alter the style of the Table of Contents
\renewcommand{\cftsecfont}{\rmfamily\mdseries\upshape}
\renewcommand{\cftsecpagefont}{\rmfamily\mdseries\upshape} % No bold!

\theoremstyle{definition}
\newtheorem{definition}{Definition}[section]

%%% END Article customizations

%%% The "real" document content comes below...

\title{Optimization Topics Report}
\author{Joshua Gunter\\  Daniel Snider}
%\date{} % Activate to display a given date or no date (if empty),
         % otherwise the current date is printed 

\begin{document}
\maketitle

\section{Introduction}

\section{Definitions}

\subsection{Linear and Integer Programming}

\subsection{Groups and Representations}

\subsection{Core Points}

In this section, we will be looking over the definitions of core points and related concepts.

\begin{definition}[Core Point]
A core point of a permutation group $G$ is a point $z \in \mathbb{Z}^n$ such that the orbit polytope $\textrm{\textbf{conv}}(Gz)$ contains no interior integer points. In other words, $\textrm{\textbf{conv}}(Gz) \cap \mathbb{Z}^n = Gz$.
\end{definition}

The optimal solution to an symmetric integer program will always be a core point of the underlying symmetry group.

\section{Methods \& Algorithms}

In order to restrict the search space for integer programming problems, we will need to have methods to compute the invariant subspaces of a symmetry and then generate corresponding quadratic constraints.

\subsection{Computing Invariant Subspaces}

There are two methods we can use to compute invariants subspaces of symmetry groups. One method relies primarily on the irreducible characters of a group representation, while the other sets up and solves a set of polynomial equations constructed from the orbits of a group. 

\subsubsection{Irreducible Characters}



\subsubsection{Solving Polynomials}

\subsection{Constructing Core Points \& Quadratic Constraints}

\subsection{Solving Integer Programs with CPLEX}

\section{Computational Results}

\section{Recommendations}

\end{document}

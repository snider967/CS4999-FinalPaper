% !TEX TS-program = pdflatex
% !TEX encoding = UTF-8 Unicode

% This is a simple template for a LaTeX document using the "article" class.
% See "book", "report", "letter" for other types of document.

\documentclass[11pt]{article} % use larger type; default would be 10pt

\usepackage[utf8]{inputenc} % set input encoding (not needed with XeLaTeX)

%%% Examples of Article customizations
% These packages are optional, depending whether you want the features they provide.
% See the LaTeX Companion or other references for full information.

%%% PAGE DIMENSIONS
\usepackage{geometry} % to change the page dimensions
\geometry{a4paper} % or letterpaper (US) or a5paper or....
% \geometry{margin=2in} % for example, change the margins to 2 inches all round
% \geometry{landscape} % set up the page for landscape
%   read geometry.pdf for detailed page layout information

\usepackage{graphicx} % support the \includegraphics command and options

% \usepackage[parfill]{parskip} % Activate to begin paragraphs with an empty line rather than an indent

%%% PACKAGES
\usepackage{booktabs} % for much better looking tables
\usepackage{amssymb}
\usepackage{amsthm}
\usepackage{amsmath}
\usepackage{listings}
\usepackage{array} % for better arrays (eg matrices) in maths
\usepackage{paralist} % very flexible & customisable lists (eg. enumerate/itemize, etc.)
\usepackage{verbatim} % adds environment for commenting out blocks of text & for better verbatim
\usepackage{subfig} % make it possible to include more than one captioned figure/table in a single float
% These packages are all incorporated in the memoir class to one degree or another...

%%% HEADERS & FOOTERS
\usepackage{fancyhdr} % This should be set AFTER setting up the page geometry
\pagestyle{fancy} % options: empty , plain , fancy
\renewcommand{\headrulewidth}{0pt} % customise the layout...
\lhead{}\chead{}\rhead{}
\lfoot{}\cfoot{\thepage}\rfoot{}

%%% SECTION TITLE APPEARANCE
\usepackage{relsize}
\usepackage{sectsty}
\allsectionsfont{\mdseries\upshape\larger} % (See the fntguide.pdf for font help)
% (This matches ConTeXt defaults)

%%% ToC (table of contents) APPEARANCE
\usepackage[nottoc,notlof,notlot]{tocbibind} % Put the bibliography in the ToC
\usepackage[titles,subfigure]{tocloft} % Alter the style of the Table of Contents
\renewcommand{\cftsecfont}{\rmfamily\mdseries\upshape}
\renewcommand{\cftsecpagefont}{\rmfamily\mdseries\upshape} % No bold!

\theoremstyle{definition}
\newtheorem{theorem}{Theorem}[section]
\newtheorem{definition}[theorem]{Definition}
\newtheorem{lemma}[theorem]{Lemma}
\newtheorem*{example}{Example}

\theoremstyle{remark}
\newtheorem*{remark}{Remark}

\newcommand{\ZZ}{\mathbb{Z}}
\newcommand{\RR}{\mathbb{R}}
\newcommand{\code}[1]{\texttt{#1}}
\newcommand{\myspan}[1]{\textrm{span} \lbrace {#1} \rbrace}

%%% END Article customizations

%%% The "real" document content comes below...

\title{Optimization Topics Report}
\author{Joshua Gunter\\  Daniel Snider}
%\date{} % Activate to display a given date or no date (if empty),
         % otherwise the current date is printed 

\begin{document}
\maketitle

%% INTRODUCTION
\section{Introduction}

Linear programming is a method of mathematical optimization, where the goal is to find the optimal value of a linear objective function over a set of variables, which are constrained by a set of linear inequalities. If all the variables must also be integers, then the problem is referred to as an integer program. While linear programs may be efficiently solved with the use of algorithms such as the Simplex Method, there is no reliably fast method for integer programming problems. In fact, they have been proven to be NP-Hard. Because of this, better methods and algorithms for solving integer programs is an active field of research.

In this study, we investigated a method to speed up the solving of a special set of symmetric integer programs. An integer program (hereafter referred to as an IP) has symmetry if the value of its objective function is invariant under permutations of the coordinate axes. More plainly, swapping around the entries of a vector $v$ describing the solution to an IP doesn't change the value of $f(v)$, where $f$ is the objective function.

When an IP is symmetric in this way we can find a set of orthogonal subspaces in $\mathbb{R}^n$, at least one of which any optimal solution must lie "close to". We can therefore restrict the search space for the optimum by setting up quadratic bounds, restricting how far away an answer can be from the aforementioned subspaces. Exactly how "close to" one of these subspaces the solution has to be is not yet known. The goal of this project is to experimentally test different possible bounds and discover what the effect of adding these bounds are on the runtime of solving IPs using standard solvers. In general, it is expected that with a tighter bound, the solver should be able to find a solution faster than with a looser bound, as there is less space to search through. By experimentally checking various bounds, we can determine whether or not this strategy would be useful once methods to find an exact bound are discovered.

This report consists of four main sections. First, definitions of terms from linear programming, group theory, and geometry required as background. Second we look at the algorithms and methods used to construct special subspaces and integer programs, and compute the additional quadratic constraints. Third we list some of the computational results found. Finally, we give some comments and recommendations for future work.

%% DEFINITIONS
\section{Definitions}

\subsection{Linear and Integer Programming}

\subsection{Groups and Representations}

% CORE POINTS
\subsection{Core Points}

In this section, we will be looking at the definition of what a core point is, and how they relate integer programs. These objects were first studied by \cite{some}, in the context of [...].

\begin{definition}[Core Point]
A core point of a permutation group $G$ is a point $z \in \mathbb{Z}^n$ such that the orbit polytope $\textrm{conv}(Gz)$ contains no interior integer points. In other words, $\textrm{conv}(Gz) \cap \mathbb{Z}^n = Gz$. The set of all core points of $G$ is called the \textbf{core set} and is denoted by $\textrm{core}(G)$.
\end{definition}

A very important result from \cite{rehn} is that these core points must lie "close to" the invariant subspaces of their group $G$. The proof of this is rather technical, so rather than give the full theorem and proof we will sketch out the general idea behind it. First, we define a function $\mu : \ZZ^n -> \RR$ which maps points $z \in \ZZ^n$ to the minimal distance of a point in the affine hull of $Gz$ from the fixed space $\textrm{Fix}(G)$. We then construct a minimal bounding ellipsoid $E$ around the orbit polytope $\textrm{conv}(Gz)$, where $z \in \textrm{core}(G)$, using results from Theorem 2.2 in \cite{barvinok/blekherman}. We then find a scaled and translated ellipsoid $E'$ constructed from $E$ such that $E' \subseteq \textrm{conv}(Gz)$, and use this to find conditions for when $E'$ (and by extension $\textrm{conv}(Gz)$ contain interior integer points. From this exploration we can find a constant bound $C(G)$ for the projection of $z$ onto any of the invariant subspaces of $G$, such that $z \in \textrm{core}(G) \implies  ||z|_{V_i}|| \leq C(G)$. If $||z|_{V_i}|| \geq C(G)$, then $E' \subseteq \textrm{conv}(Gz)$ will contain an integer point inside of it, and therefore $z$ can not be a core point.

A natural question is what these points have to do with solving integer programs. Before that question can be answered, let us state the precise definition of a symmtery of an integer programming problem. 

\begin{definition}[IP Symmetry]
A \textbf{symmetry} of an integer program with a feasible region $P \subset \RR^n$ and objective function $f : \ZZ^n \rightarrow \RR$ is a permutation $g \in S_n$ such that $gx \in P$ and $f(x) = f(gx)$ for all $x \in P$. The group of all these permutations is the symmetry group $G \leq S_n$ of the integer program.
\end{definition}

Now we can state the result connecting these two concepts of core points and integer programs, Theorem 7.3 from \cite{rehn}.

\begin{theorem}
Let $\min_{x \in C \cap \ZZ^n} f(x)$ be a convex integer optimization problem, with a convex function $f$ and a convex set $C \subseteq \mathbb{R}^n$ and a symmetry group $G$. Then $\min_{x \in C \cap \ZZ^n} f(x) = \min_{x \in C \cap \textrm{core}(G)} f(x)$.
\end{theorem}

The idea behind the proof of this theorem is that if the optimal solution $x \in C \cap \ZZ^n$ is not a core point, then we can find some integral points inside of $\textrm{conv}(Gx)$. By choosing an $x'$ from within the integer points $\textrm{conv}(Gx) \cap \ZZ^n$ that has minimal norm, we are guaranteed that $x'$ is a core point for $G$. Since $x' \in \textrm{conv}(Gz)$, $x'$ is a convex combination of permutations of $x$, so $x' = tx + (1-t)(gx)$ for some $t \in [0,1], g \in G$. Because $f$ is convex, $f(tx + (1-t)(gx)) \leq tf(x) + (1-t)f(gx) \implies f(x') \leq f(x)$, as $f(x) = f(gx)$. Therefore $x'$ is an optimal solution for the integer program with $x' \in \textrm{core}(G)$.

Due to this theorem, we can find an optimal solution to any integer program with a symmetry by only looking at the core points of the symmetry group. In some cases, there are a finite number of core points, and we can devise algorithms based on enumerating core points and searching for an optimal solution. However in this project we focused on a set of groups containing infinite core points, in which case enumeration of core points won't help us any. Instead, armed with our additional constraints on the norm of the projection of core points on the invariant subspaces of the group, we can restrict the size of the feasible region when searching for solutions to the integer program. This is the extent of the knowledge we will need about core points for this project.

%% METHODS
\section{Methods \& Algorithms}

Now that we know that we can add additional constraints to reduce the size of the feasible region, all we need are algorithms to actually compute all the required data. First, we will examine two methods to construct invariant subspaces of symmetry groups, one method using information from character theory, the second using results about group actions in relation to invariant subspaces. Next, we will look at how we can actually compute a set of core points of group, and use these to construct difficult integer programs to test the effectiveness of this method on. Finally, we will cover how to compute the new quadratic constraints that we can use to solve these problems faster.

% INVARIANT SUBSPACES
\subsection{Computing Invariant Subspaces}

% METHOD 1
\subsubsection{Irreducible Characters}

This method was the primary one used during the project, and makes use of results found in character theory. Many formulas dealing with representations of groups can be transformed into formulas involving their characters, which are much easier to work with computationally.

\begin{definition}[Character] 
The \textbf{character} $\chi_\rho : G \rightarrow \mathbb{C}$ of a representation $\rho : G \rightarrow GL_n(\mathbb{C})$ is defined as $\textrm{tr}(\rho(g))$ for all $g \in G$.
\end{definition}

Similarly to how representations can be decomposed into a direct sum of irreducible representations, the character of a representation can be decomposed into a linear combination of irreducible characters. This means we can write any character of a group G as

	$$ \chi = m_1\chi_1 + m_2\chi_2 + ... + m_k\chi_k, $$

where $m_i \in \mathbb{Z}_{\geq 0}$, and $\chi_i$ is the character of the $i$th irreducible representation of $G$. We can compute each $m_i$ by taking an inner product of the characters, using the formula:

\begin{equation}
m_i = \frac{1}{|G|} \sum_{g \in G} \chi_i(g) \chi(g^{-1}).
\end{equation}

From this, we can see that as long as we know the values of the irreducible characters, we can easily compute what the decomposition is for any character.

The decomposition of a character $\chi$ into a sum of irreducible characters corresponds to the decomposition of the related representation $\rho$ into a direct sum of subrepresentations. We can rewrite $\rho$ into

	$$ \rho = \rho_1^{(m_1)} \oplus \rho_2^{(m_2)} \oplus ... \rho_k^{(m_k)}. $$

Each of these subrepresentations $\rho_i^{(m_i)}$ corresponds to an invariant subspace $V_i$, which is what we're really interested in. In \cite{char}, we can find the following formula for the projection of $\mathbb{C}^n$ onto $V_i$:

\begin{equation} \label{eq:projection}
P_i = \frac{m_i\chi_i(1)}{|G|} \sum_{g \in G} \chi_i(g^{-1})\rho(g).
\end{equation}

The column space of $P_i$ is then a basis for $V_i$. At this point we have all of the basic tools necessary for computation of invariant subspaces. However these formulas do have a weakness: as they require summing over the all of the group elements, the runtime required can be exorbitant for larger groups. We can make our lives a little easier by doing some computations with conjugacy classes of groups, along with a couple of additional small optimizations.

\begin{definition}[Conjugacy Classes]
For a group $G$, two elements $a, b \in G$ are said to be \textbf{conjugate} if $a = gbg^{-1}$ for some $g \in G$. A \textbf{conjugacy class} is a subset of $G$ containing elements that conjugate with one another. The conjugacy class containing an element $g \in G$ is denoted by $Cl(g)$.
\end{definition}

This relation can easily be seen to be reflexive, symmetric, and transitive, and is therefore an equivalence relation. As such, we can partition $G$ in a set of disjoint conjugacy classes. This is very useful to us because characters are constant over a conjugacy class, so by dealing with conjugacy classes we can avoid computations summing over every group element. Instead we can sum over the conjugacy classes whenever we are only dealing with characters. We can perform an additional optimization making use of the fact that $\chi(g^{-1}) = \overline{\chi(g)}$, where $\overline z$ denotes the complex conjugate of $z \in \mathbb{C}$. This allows us to compute $\chi(g^{-1})$ without needing to know what $g^{-1}$ actually equals. We can now re-express the previous formula for computing the coefficients of the irreducible characters in the decomposition of a character by

\begin{equation} \label{eq:chars}
m_i = \frac{1}{|G|} \sum_{Cl(g) \subseteq G} |Cl(g)| \chi_i(g) \overline{\chi(g)}.
\end{equation}

This is the equation we will actually be implementing. Unfortunately as for the second part of finding invariant subspaces, computing the projection matrices $P_i$, due to the fact that the representation itself is being used in the formula we will not be able to make use of conjugacy classes to save time, as representations are definitely not constant over conjugacy classes.

All of these computations were performed using GAP, which has excellent support for computations with characters. Character tables containing information about characters of a group $G$ can be obtained with the function \code{CharacterTable(G)}, with the values for all irreducible characters of a group found in the attribute \code{Irr}. The following code block is the GAP implementation of (\ref{eq:chars}).

\begin{lstlisting}[frame = single]
tbl := CharacterTable(G);;
irrs := Irr(tbl);;
reg := RegularCharacters(G,dim);;

m := List(irrs, i -> 0);;
for i in [1..Size(irrs)] do
    m[i] := Sum([1..Size(SizesConjugacyClasses(tbl))],
                j -> SizesConjugacyClasses(tbl)[j]*irrs[i][j]
		    *ComplexConjugate(reg[j]) ) / Order(G);
od;
\end{lstlisting}

Here \code{SizesConjugacyClasses} is a list of the cardinalities of the conjugacy classes of $G$. Next we have the implementation of (\ref{eq:projection}) in GAP:

\begin{lstlisting}[frame = single]
P := List(irrs, i -> NullMat(dim,dim));;
for i in [1..Size(irrs)] do
    if m[i] <> 0 then
        for g in G do
            irr_index := Position(cclasses,ConjugacyClass(G,g));
            reg_rep := PermutationMat(g,dim);
            P[i] := P[i] + ComplexConjugate(irrs[i][irr_index])
			*reg_rep;
        od;
        P[i] := m[i]*irrs[i][1]/Order(G) * P[i];
    fi;
od;
\end{lstlisting}

\code{Position(Classes, C)} returns the index of a conjugacy class $C$ in the list of conjugacy classes.

Once we've computed the projection matrices $P_i$, all that's left to do is compute their column spaces. A simple method to achieve this in GAP is to transpose the matrix with \code{TransposedMat}, and then find the row space of that using the \code{BaseMat} function. After that, we will have a list of bases of invariant subspaces for a group.

\begin{example}
Consider the permutation group $G = \langle (12), (34) \rangle \cong S_2 \times S_2 $, with the regular representation $\rho : G \rightarrow GL_4(\mathbb{C})$ mapping permutations to permutation matrices. We have $\rho(()) = I$, 
\[
\rho((12)) =
	\begin{bmatrix}
	0 & 1 & 0 & 0 \\
	1 & 0 & 0 & 0 \\
	0 & 0 & 1 & 0 \\
	0 & 0 & 0 & 1
	\end{bmatrix}
,\
\rho((34)) =
	\begin{bmatrix}
	1 & 0 & 0 & 0 \\
	0 & 1 & 0 & 0 \\
	0 & 0 & 0 & 1 \\
	0 & 0 & 1 & 0
	\end{bmatrix}
,\
\rho((12)(34)) =
	\begin{bmatrix}
	0 & 1 & 0 & 0 \\
	1 & 0 & 0 & 0 \\
	0 & 0 & 0 & 1 \\
	0 & 0 & 1 & 0
	\end{bmatrix},
\]
from which it is easy to see that
\[
\chi(()) = 4,\ \chi((12)) = 2,\ \chi((34)) = 2, \textrm{and}\ \chi((12)(34)) = 0.
\]
We can write this function compactly as an array indexed by group elements $\chi = \lbrack 4, 2, 2, 0 \rbrack$. By doing a quick calculation with GAP, we can get that the irreducible characters of this group are
\[
\chi_1 = \lbrack 1, 1, 1, 1 \rbrack, \
\chi_2 = \lbrack 1, -1, -1, 1 \rbrack
\]
\[
\chi_3 = \lbrack 1, -1, 1, -1 \rbrack, \
\chi_4 = \lbrack 1, 1, -1, -1 \rbrack.
\]

Now we can compute the coefficients of the irreducible character decomposition $\chi = m_1\chi_1 + m_2\chi_2 + m_3\chi_3 + m_4\chi_4$:
\[ m_1 = (1 \cdot 4 + 1 \cdot 2 + 1 \cdot 2 + 1 \cdot 0)/4 = 8/4 = 2 \]
\[ m_2 = (1 \cdot 4 + (-1) \cdot 2 + (-1) \cdot 2 + 1 \cdot 0)/4 = 0/4 = 0 \]
\[ m_3 = (1 \cdot 4 + (-1) \cdot 2 + 1 \cdot 2 + (-1) \cdot 0)/4 = 4/4 = 1 \]
\[ m_4 = (1 \cdot 4 + 1 \cdot 2 + (-1) \cdot 2 + (-1) \cdot 0)/4 = 4/4 = 1 \]

Therefore $\chi = 2\chi_1 + \chi_3 + \chi_4$. We can now apply Equation \eqref{eq:projection} to find e.g. $P_1$:

\[
P_1 = \frac{m_1\chi_1(1)}{4}
	\left(
	\begin{bmatrix}
	1 & 0 & 0 & 0 \\
	0 & 1 & 0 & 0 \\
	0 & 0 & 1 & 0 \\
	0 & 0 & 0 & 1
	\end{bmatrix}
	+
	\begin{bmatrix}
	0 & 1 & 0 & 0 \\
	1 & 0 & 0 & 0 \\
	0 & 0 & 1 & 0 \\
	0 & 0 & 0 & 1
	\end{bmatrix}
	+
	\begin{bmatrix}
	1 & 0 & 0 & 0 \\
	0 & 1 & 0 & 0 \\
	0 & 0 & 0 & 1 \\
	0 & 0 & 1 & 0
	\end{bmatrix}
	+
	\begin{bmatrix}
	0 & 1 & 0 & 0 \\
	1 & 0 & 0 & 0 \\
	0 & 0 & 0 & 1 \\
	0 & 0 & 1 & 0
	\end{bmatrix},
	\right)
\]
\[
P_1 = \frac{2}{4} \cdot
	\begin{bmatrix}
	2 & 2 & 0 & 0 \\
	2 & 2 & 0 & 0 \\
	0 & 0 & 2 & 2 \\
	0 & 0 & 2 & 2 
	\end{bmatrix}
	= \begin{bmatrix}
	1 & 1 & 0 & 0 \\
	1 & 1 & 0 & 0 \\
	0 & 0 & 1 & 1 \\
	0 & 0 & 1 & 1 
	\end{bmatrix}
\]
Taking the column space of $P_1$, we find that $V_1 = \myspan{(1, 1, 0, 0), (0, 0, 1, 1)}$. By the same method we get $V_2 = \myspan{(1, -1, 0, 0)}$ and $V_3 = \myspan{(0, 0, -1, 1)}$. Therefore $\myspan{(1, 1, 0, 0), (0, 0, 1, 1)} \oplus \myspan{(1, -1, 0, 0)} \oplus \myspan{(0, 0, -1, 1)}$ is a decomposition of $\mathbb{C}^4$ into $G$-invariant subspaces.
\end{example}

As stated earlier, this was the primary method used in this project, as aside from the symmetric and alternating groups, most groups we looked at had relatively small orders (less than a few hundred thousand). With larger groups, another method is required to solve for their invariant subspaces.

% METHOD 2
\subsubsection{Solving Polynomials}

An alternative method 

%% CORE POINTS + IPS
\subsection{Constructing Core Points \& Integer Programs}

In order to test out the effectiveness of the added constraints, we'll need to construct integer programs with symmetries that are difficult to solve with standard solvers. For this purpose, we can use the orbit polytopes of core points of groups. Since the vertices of these polytopes are permutations of a vector, they will have the corresponding group symmetry. In addition, as the orbit polytope of a core point contains no integer points aside from the vertices, if we can cut off these vertices then the integer program corresponding to this polytope will be infeasible. This will ensure that a solver won't by chance rapidly solve the problem, and will have to fully explore the branch and bound tree.

So to find integer programs we can use to test our methods, we'll need to find core points. This is non-trivial, but from \cite{rehn} we have some tools for this. In particular, there is a certain class of groups (those whose representations can be decomposed into at least two rational subspaces, excluding the fixed space) which have an infinite number of non-isomorphic core points, along with a simple method to generate them. Much of the reasoning depends on points that have globally minimal projection.

\begin{definition}[Globally Minimal Projection]
A point $z \in \ZZ$ has \textbf{globally minimal projection} with respect to a subspace $V \subseteq \mathbb{R}^n$ if $||z|_V|| \leq ||z'||_V||$ for all $z' \in \textrm{aff}(Gz) \cap \mathbb{Z}^n$.
\end{definition}

So if a point has globally minimal projection onto a subspace, it is the closest point in the affine hull of $Gz$ to that subspace.

% QUADRATICS

\subsection{Quadratic Constraints}

From 

\subsection{Solving Integer Programs with CPLEX}

\section{Computational Results}

\section{Recommendations}

\end{document}
